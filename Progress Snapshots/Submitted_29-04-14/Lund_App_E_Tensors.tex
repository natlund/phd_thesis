% Finished July 18, 2013
% Put into format of VUW Thesis on March 20, 2014.
% Cut out of Chapter 6 and formed into Appendix E on Monday, 24 March 2014.

\documentclass[12pt, a4paper, twoside, openright]{book}

\usepackage{vuwthesis} % sets up some local things, mostly the front page

\setlength{\intextsep}{12pt} % set space above and below in-line float
\setlength{\abovecaptionskip}{0pt} % set space between figure and caption.


\usepackage{amssymb, amsmath}
%\usepackage{mathtools}
\usepackage{tikz}
\usetikzlibrary{calc}

\newcommand{\beff}{\ensuremath{b_{\mathrm{eff}}}}
\newcommand{\bhom}{\ensuremath{b_{\mathrm{hom}}}}

%\usepackage{marvosym}

\usepackage{etoolbox}
\newtoggle{compilealone}
\toggletrue{compilealone}

\title{Appendix E: Tensor Identities}
\author{Nat Lund}

\begin{document}
\chapter{Tensor Identities}\label{C:tensors}

In this Appendix, we introduce the double dot product of two tensors, and work with the velocity gradient tensor to ultimately derive the tensor identity
\begin{equation*}
\nabla^2 \vec{u} \cdot \vec{g} = 
\nabla \cdot ( (\nabla \vec{u} + \nabla \vec{u}^T) \cdot \vec{g})
- 2 \mathbf{E}(\vec{u}):\mathbf{E}(\vec{g})
\end{equation*}
for use in Chapter 6.


\subsection{Tensor Double Dot Product}

The double dot product of two tensors, also known as the Frobenius inner product, is a generalization of the vector inner product:

\begin{equation}
A = 
\begin{bmatrix}
a & b \\
c & d
\end{bmatrix}
, \quad Z = 
\begin{bmatrix}
x & y \\
z & w
\end{bmatrix}
, \quad
A:Z = ax + by + cz + dw
\end{equation}

As expected, addition distributes over this form of multiplication:
\begin{equation}
(A + B):(Z + W) = A:Z + A:W + B:Z + B:W
\end{equation}

\clearpage
\subsection{Tensor Vector Divergence Identity}

For a tensor $T$ and vector $\vec{g}$:

\begin{equation}
T:\nabla \vec{g} = T_{11}\partial_x g_x + T_{12}\partial_x g_y + T_{21}\partial_y g_x
+ T_{22} \partial_y g_y
\end{equation}
and
\begin{equation}
\nabla \cdot T = 
\begin{bmatrix}
\partial_x & , & \partial_y
\end{bmatrix}
\begin{bmatrix}
T_{11} & T_{12} \\
T_{21} & T_{22}
\end{bmatrix} =
\begin{bmatrix}
\partial_x T_{11} + \partial_y T_{21} &,& \partial_x T_{12} + \partial_y T_{22}
\end{bmatrix}
\end{equation}
so that
\begin{equation}
(\nabla \cdot T) \cdot \vec{g} = 
g_x \partial_x T_{11} + g_x \partial_y T_{21} + g_y \partial_x T_{12}
 + g_y \partial_y T_{22}
\end{equation}
 Furthermore
\begin{equation}
T \cdot \vec{g} = 
\begin{bmatrix}
T_{11} & T_{12} \\
T_{21} & T_{22}
\end{bmatrix}
\begin{bmatrix}
g_x \\
g_y
\end{bmatrix} = 
\begin{bmatrix}
T_{11} g_x + T_{12} g_y \\
T_{21} g_x + T_{22} g_y
\end{bmatrix}
\end{equation}
Therefore
% "What do the underbraces mean?".   Easier to just rearrange terms into groups inside square brackets.
%\begin{align*}
%\nabla \cdot (T \cdot \vec{g}) & = 
%\partial_x (T_{11} g_x) + \partial_x (T_{12} g_y) +
%\partial_y (T_{21} g_x) + \partial_y (T_{22} g_y) \\
% & = g_x \partial_x T_{11} + \underbrace{T_{11} \partial_x g_x} +
%     g_y \partial_x T_{12} + \underbrace{T_{12} \partial_x g_y} +
%     g_x \partial_y T_{21} + \underbrace{T_{21} \partial_y g_x} +
%     g_y \partial_y T_{22} + \underbrace{T_{22} \partial_y g_y} \\
% & = T:\nabla \vec{g} + (\nabla \cdot T) \cdot \vec{g}
%\end{align*}
\begin{align}
\nabla \cdot (T \cdot \vec{g}) & = 
\partial_x (T_{11} g_x) + \partial_x (T_{12} g_y) +
\partial_y (T_{21} g_x) + \partial_y (T_{22} g_y) \\
 & = g_x \partial_x T_{11} + T_{11} \partial_x g_x +
     g_y \partial_x T_{12} + T_{12} \partial_x g_y \\
 & \quad +    g_x \partial_y T_{21} + T_{21} \partial_y g_x +
     g_y \partial_y T_{22} + T_{22} \partial_y g_y \\
 & =\left[ T_{11} \partial_x g_x  +  T_{12} \partial_x g_y +
           T_{21} \partial_y g_x  +  T_{22} \partial_y g_y \right] \\
 & \quad + \left[ g_x \partial_x T_{11}  +  g_x \partial_y T_{21} +
           g_y \partial_x T_{12}  +  g_y \partial_y T_{22} \right] \\
 & = T:\nabla \vec{g} + (\nabla \cdot T) \cdot \vec{g}
\end{align}


We have shown:
\begin{equation}
\nabla \cdot (T \cdot \vec{g}) =  T:\nabla \vec{g} + (\nabla \cdot T) \cdot \vec{g}
\end{equation}

\clearpage
\subsection{Application to Velocity Gradient Tensor}

Substituting $T = \nabla \vec{u}$ in the identity gives:
\begin{equation}
\nabla \cdot (\nabla \vec{u} \cdot \vec{g}) = 
\nabla \vec{u}:\nabla \vec{g} + (\nabla \cdot \nabla \vec{u}) \cdot \vec{g}
\end{equation}
and the `vector Laplacian' is defined:
\begin{equation}
\nabla \cdot \nabla \vec{u} =
\begin{bmatrix}
\partial_x \partial_x u + \partial_y \partial_y u &,& \partial_x \partial_x v + \partial_y \partial_y v
\end{bmatrix} =
\begin{bmatrix}
\nabla^2 u &,& \nabla^2 v
\end{bmatrix}
= \nabla^2 \vec{u}
\end{equation}
So
\begin{equation}
\nabla \cdot (\nabla \vec{u} \cdot \vec{g}) = 
\nabla \vec{u}:\nabla \vec{g} + \nabla^2 \vec{u} \cdot \vec{g}
\end{equation}

Similarly for the transpose of the velocity gradient tensor:
\begin{equation}
\nabla \cdot (\nabla \vec{u}^T \cdot \vec{g}) = 
\nabla \vec{u}^T:\nabla \vec{g} + (\nabla \cdot \nabla \vec{u}^T) \cdot \vec{g}
\end{equation}
Now, however, the last term vanishes:
\begin{align}
\nabla \cdot \nabla \vec{u}^T & =
\begin{bmatrix}
\partial_x \partial_x u + \partial_x \partial_y v &,&
 \partial_y \partial_x u + \partial_y \partial_y v
\end{bmatrix} \\
 & =
\begin{bmatrix}
\partial_x ( \partial_x u + \partial_y v ) &,&
\partial_y ( \partial_x u + \partial_y v )
\end{bmatrix} \\
 & = 
\begin{bmatrix}
\partial_x ( \nabla \cdot \vec{u}) &,&
\partial_y ( \nabla \cdot \vec{u})
\end{bmatrix} \\
 & = 
 \begin{bmatrix}
 0 &,& 0
 \end{bmatrix}
\end{align}
since we assume the fluid is \textbf{incompressible}, so $\nabla \cdot \vec{u} = 0$ everywhere.

Thus
\begin{equation}
\nabla \cdot (\nabla \vec{u}^T \cdot \vec{g}) = \nabla \vec{u}^T:\nabla \vec{g}
\end{equation}

\clearpage
\subsection{Deformation Rate Tensor Identity}

Extending our notation slightly to include vector fields other than $\vec{u}$, recall that the deformation rate tensor is:
\begin{equation}
\mathbf{E}(\vec{u}) = \frac{\nabla \vec{u} + \nabla \vec{u}^T }{2}
\end{equation}
so that:
\begin{equation}
2 \mathbf{E}(\vec{g}) = \nabla \vec{g} + \nabla \vec{g}^T
\end{equation}

Then the double dot product of two such tensors is:
\begin{align}
2 \mathbf{E}(\vec{u}):2 \mathbf{E}(\vec{g}) & =
(\nabla \vec{u} + \nabla \vec{u}^T ):(\nabla \vec{g} + \nabla \vec{g}^T ) \\
4 \mathbf{E}(\vec{u}): \mathbf{E}(\vec{g}) & =
   \nabla \vec{u}:\nabla \vec{g} + \nabla \vec{u}:\nabla \vec{g}^T
 + \nabla \vec{u}^T:\nabla \vec{g} + \nabla \vec{u}^T: \nabla \vec{g}^T
\end{align}
Now, transposition affects the double dot product such that \\
 $\nabla \vec{u}:\nabla \vec{g} =\nabla \vec{u}^T: \nabla \vec{g}^T$ 
 and $ \nabla \vec{u}^T:\nabla \vec{g} = \nabla \vec{u}:\nabla \vec{g}^T $ , so
\begin{align}
4 \mathbf{E}(\vec{u}): \mathbf{E}(\vec{g}) & =
   2 \nabla \vec{u}:\nabla \vec{g} + 2 \nabla \vec{u}^T:\nabla \vec{g} \\
2 \mathbf{E}(\vec{u}): \mathbf{E}(\vec{g}) & =
   \nabla \vec{u}:\nabla \vec{g} + \nabla \vec{u}^T:\nabla \vec{g}   
\end{align}

\vspace{1em}
Finally,
\begin{align}
\nabla \cdot ( (\nabla \vec{u} + \nabla \vec{u}^T) \cdot \vec{g}) & =
\nabla \cdot ( \nabla \vec{u} \cdot \vec{g} + \nabla \vec{u}^T \cdot \vec{g}) \\
  & = \nabla \cdot ( \nabla \vec{u} \cdot \vec{g}) +
      \nabla \cdot ( \nabla \vec{u}^T \cdot \vec{g}) \\
  & = \nabla^2 \vec{u} \cdot \vec{g} + \nabla \vec{u}:\nabla \vec{g}
      + \nabla \vec{u}^T:\nabla \vec{g} \\
  & = \nabla^2 \vec{u} \cdot \vec{g} + 2 \mathbf{E}(\vec{u}):\mathbf{E}(\vec{g})
\end{align}
Therefore:
\begin{equation}
\nabla^2 \vec{u} \cdot \vec{g} = 
\nabla \cdot ( (\nabla \vec{u} + \nabla \vec{u}^T) \cdot \vec{g})
- 2 \mathbf{E}(\vec{u}):\mathbf{E}(\vec{g})
\end{equation}

\iftoggle{compilealone}
    {
    \bibliography{Lund_Thesis.bib}
    \bibliographystyle{plain}
    }

\end{document}
